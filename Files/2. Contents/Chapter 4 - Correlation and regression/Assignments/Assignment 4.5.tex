%%%%%%%%%%%%%%%%%%%%%%%%%%%%%%%%%%%%%%%%%%%%%%%%%%%%%%%%%%%%%%%%%%%%%%%%%%%
% Assignment 4.5: Making predictions using linear regression in R
%%%%%%%%%%%%%%%%%%%%%%%%%%%%%%%%%%%%%%%%%%%%%%%%%%%%%%%%%%%%%%%%%%%%%%%%%%%

\rassignment{Assignment 4.5: Making predictions using linear regression in R}

The manager now wants to know what will happen to the number of milk cartons that she has to throw away when she lowers the price of a milk carton by 30 cents, from \$1.00 to \$0.70. Currently, the supermarket throws away 4 cartons of milk each day. \\

\question{
    4.5 a
}{
    Create a new data frame that has only one column that includes the new value for the price of milk (the column has to be named exactly the same as in \rcode{dataset7}).
}

\rcodeanswersmall{4.5a}

The \rcode{predict()} function can be used to predict new data according to a \concept{linear model}. \\

\question{
    4.5 b
}{
    Use the \rcode{predict()} function to predict the number of milk cartons that the supermarket will have to throw away when the \rcode{Price} is \$0.70.
}

\rcodeanswertiny{4.5b}

\emptyanswerbox{
    4.5b
}{
    Predicted value: \shortanswerline
}

\question{
    4.5 c
}{
    Confirm this estimate by writing out the regression equation of the \concept{linear model} from assignment 4.4e and filling in the new value of the price of milk. 
}

\emptyanswerbox{
    4.5c
}{
    Predicted value: \mediumanswerline
}

The \rcode{predict()} function can also be used to construct a \concept{confidence interval} for the predicted number of cartons thrown away by using \rcode{interval = \textquotesingle prediction\textquotesingle}. \\

\clearpage % Page break

\question{
    4.5 d
}{
    Create a 90\% \concept{confidence interval} for the predicted number of milk cartons thrown away. 
}

\rcodeanswertiny{4.5d}

\question{
    4.5 e
}{
    When the manager lowers her price from \$1.00 to \$0.70, will the supermarket throw away fewer cartons of milk? Incorporate the 90\% \concept{confidence interval} for the predicted value in your answer.
}

\emptyanswerbox{
    4.5e
}{
    The supermarket \textbf{will} / \textbf{will not} throw away fewer cartons of milk.
    \answerskip
    Explanation:
    \answerskip
    \rule{\textwidth}{0.4pt}
    \answerbreak
    \rule{\textwidth}{0.4pt}
}

\clearpage % Page break