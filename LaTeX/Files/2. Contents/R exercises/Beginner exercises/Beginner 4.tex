\subsection{Object types: Vectors}

\beginnerexercise{
    Use the \rcode{c()} function (or \rcode{:}) to create the following vector: 
    \begin{center}
        \rcode{-2 -1 0 1 2 3 4 5 6 7 8}
    \end{center}
    \\
    What is the length of this vector? 
}

\beginnerexercise{
    Make a vector that starts at -5 and goes to 5 in steps of 0.5. Can you find out more ways to construct this vector than by using the \rcode{c()} function?
}

\beginnerexercise{
    Create the following vector: 
    \begin{center}
        \rcode{13 15 17 19 21 23 25 27 29 31 33}
    \end{center}
}

\beginnerexercise{
    Run the following code in \texttt{R}: \\
    \\
    \codeblock{a <- 1:5 \\
    b <- -a}
    Combine the vectors \rcode{a} and \rcode{b} into one vector.
}

\beginnerexercise{
    What is the result of \rcode{rep(2, 10)}? What function argument receives the value 10?
}

\beginnerexercise{
    Use the \rcode{rep()} function to create the following vector: 
    \begin{center}
        \rcode{3 3 3 4 4 4 5 5 5 6 6 6 7 7 7}
    \end{center}
}

\beginnerexercise{
    Suppose you want to make the vector \rcode{"a" "a" "b" "b" "c" "c" "d" "d" "e" "e"} with the \rcode{rep()} function but \rcode{rep(letters[1:5], 2)} does not work. What do you have to change in this command?
}

\beginnerexercise{
    The following code gives the first 10 uneven numbers. What is the sum of these numbers? \\
    \\ 
    \codeblock{(1:10) * 2 - 1}
}

\beginnerexercise{
    Make a vector with the first 10 even numbers. Next, make a vector with the first 10 numbers divided by 5. Finally, create the vector \rcode{5 8 11 14 17 20 23 26 29 32}.
}

\beginnerexercise{
    \rcode{logical(5)} makes a logical vector of length 5. How can you make the same vector with the \rcode{vector()} function?
}

\beginnerexercise{
    Run the following code in \texttt{R}: \\
    \\
    \codeblock{paste(rep(c(\textquotesingle a\textquotesingle, \textquotesingle b\textquotesingle), each = 5), 1:5, sep = \textquotesingle .\textquotesingle)}
    Next, create the following vector: \\
    \begin{center}
        \rcode{"x1m" "x1f" "x2m" "x2f" "y1m" "y1f" "y2m" "y2f"}
    \end{center}
}

\clearpage % Page break

\beginnerexercise{
    Run the following code in \texttt{R}: \\
    \\
    \codeblock{s <- c(5, 7, 2, 8)}
    First sort \rcode{s} from highest to lowest. What option in the \rcode{sort()} function do you use? Next, sort \rcode{s} from lowest to highest. Why don't you have to use the same option here?
}

\beginnerexercise{
    Run the following code in \texttt{R} and explain why the second line gives a warning: \\
    \\
    \codeblock{1:10 + 1:2 - 1\\
    1:10 + 1:3 - 1}
}

\clearpage