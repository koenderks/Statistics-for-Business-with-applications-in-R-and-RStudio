\subsection{Creating and removing objects in the environment}

\beginnerexercise{
    Run the following code in \texttt{R}: \\
    
    \codeblock{a = b = 1\\
a = 2}

    What happened? How do you know (which command)? How many objects did you create in your environment? Try assigning the values using the \rcode{<-} operator. Does the result differ from the result when you are using \rcode{=} to create the variables? 
}   

\beginnerexercise{
    Remove the variables\rcode{a} and \rcode{b} from your environment using the \rcode{rm()} function.
}

\beginnerexercise{
    Run the following code in \texttt{R}: \\
    \\
    \codeblock{apples <- 5; pears <- 3; pineapples <- 6}
    Then try the following code: \\
    \\
    \codeblock{apples + pineapples \\
    apples + Pears
    }
    Why can't you add \rcode{apples} and \rcode{Pears}?
}

\beginnerexercise{
    Use \texttt{R} to compute the square root of 81 and store the result in \rcode{t1}. 
}

\beginnerexercise{
    Use \texttt{R} to compute 81 to the power a half and store the result in \rcode{t2}. 
}

\beginnerexercise{
    Use the \rcode{==} operator to check whether the contents of \rcode{t1} and \rcode{t2} are the same.
}

\clearpage % Page break