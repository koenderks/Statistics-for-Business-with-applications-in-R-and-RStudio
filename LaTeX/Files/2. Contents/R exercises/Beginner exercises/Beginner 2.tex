\subsection{Working directory and help functionality}

\beginnerexercise{
    With the command \rcode{getwd()} you can find the current working directory of the \texttt{R} process. Explain what the working directory means and why it is important.
}

\beginnerexercise{
    Change the working directory of the \texttt{R} session to a folder of your preference. What function do you use? How can you check whether your change has worked?
}

\beginnerexercise{
    Run the following code in \texttt{R}: \\
    \\
    \codeblock{demo()}
    What does this code do? How do you run the demo \rcode{persp} (in package \rcode{graphics})?
}

\beginnerexercise{
    Run the following code in \texttt{R}: \\
    \\
    \codeblock{a <- c(1, 6, 7, 8, 9, NA) \\
    mean(a)}
    This gives \rcode{NA}, why? Check \rcode{?mean} to find out what arguments the \rcode{mean()} function takes as input. Can you find a way to compute the mean with the missing value removed?
}

\beginnerexercise{
    According to Google there exists an R-function called \rcode{mvrnorm()}. Typing \rcode{mvrnorm} or \rcode{?mvrnorm}, however, gives an error. Why? Which library do you have to load first? How?
}

\beginnerexercise{
    What does \rcode{\%\%} do? Can you ask help with \rcode{?\%\%}, like in \rcode{?mean}? If not, how? What is \rcode{\%\%} doing?
}

\beginnerexercise{
    The \rcode{cor()} function computes correlations. As an example, run the following code in \texttt{R}: \\
    \\
    \codeblock{cor(c(1, 2, 3, 4), c(1, 4, 7, 15))}
    How can you find out (which function) whether this correlation is statistically significant? How can you find such a function? What is the confidence interval of the correlation? How can you find all functions that have 'cor' in their name? 
}

\clearpage