\setcounter{section}{2}
\setcounter{subsection}{5}
\setcounter{question}{0}

%%%%%%%%%%%%%%%%%%%%%%%%%%%%%%%%%%%%%%%%%%%%%%%%%%%%%%%%%%%%%%%%%%%%%%%%%%%
% Assignment 2.5: Creating a box plot in R
%%%%%%%%%%%%%%%%%%%%%%%%%%%%%%%%%%%%%%%%%%%%%%%%%%%%%%%%%%%%%%%%%%%%%%%%%%%

\rassignment{Creating a box plot in R}

For this next assignment, let’s return to the \rcode{swiss} data set. More specifically, you will now have to focus on the values that are stored in the separate variable \rcode{agriculture}. \\

\question{
    Find out the \concept{minimum}, \concept{lower quartile}, \concept{median}, \concept{upper quartile}, and \concept{maximum} of the \rcode{agriculture} variable.
}

\rcodeanswersmall

\emptyanswerbox{
    Minimum: \quad \shortanswerline \quad Upper quartile: \shortanswerline
            \answerskip
            Median: \quad\hspace*{7pt} \shortanswerline \quad Lower quartile: \shortanswerline
            \answerskip
            Maximum: \quad \shortanswerline
}

\question{
    Create a box plot of the \rcode{agriculture} variable.
}

\rcodeanswertiny

\question{
    Run the following code in \texttt{R} and explain the table output. How does the code work?
}

\codeblock{educationLevel <- rep(\textquotesingle2.Medium\textquotesingle, 47)\\
educationLevel[education <= 6] = \textquotesingle1.Low\textquotesingle\\
educationLevel[education >= 12] = \textquotesingle3.High\textquotesingle\\
table(educationLevel)
}

\twolineanswerbox

\clearpage % Page break

\question{
    Create a box plot using the \texttt{R} code below and explain what you see.
}

\codeblock{boxplot(agriculture {\raise.17ex\hbox{$\scriptstyle\sim$}} educationLevel)}

\twolineanswerbox

\clearpage % Page break