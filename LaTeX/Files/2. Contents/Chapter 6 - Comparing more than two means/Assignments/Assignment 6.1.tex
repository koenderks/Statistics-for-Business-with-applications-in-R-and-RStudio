%%%%%%%%%%%%%%%%%%%%%%%%%%%%%%%%%%%%%%%%%%%%%%%%%%%%%%%%%%%%%%%%%%%%%%%%%%%
% Assignment 6.1: Implementing a t-test as a regression in R
%%%%%%%%%%%%%%%%%%%%%%%%%%%%%%%%%%%%%%%%%%%%%%%%%%%%%%%%%%%%%%%%%%%%%%%%%%%

\rassignment{Assignment 6.1: Implementing a t-test as a regression in R}

Many of the statistical tests that we have seen are actually equivalent to a specific form of \concept{linear regression}. To understand how a \concept{t-test} can be implemented as a \concept{linear regression}, let’s look at an example. Suppose you work in advertising and show four groups of people an advertisement, where the only difference is in the color/position of the eyes of the model (blue eyes, brown eyes, green eyes, or downward-looking eyes), and you ask them how they rate your brand after seeing this advertisement. For this assignment, we will use the \dataset{eyeColor.csv} data file that contains 222 participants’ ratings for one of the four groups. As a first question, you want to assess whether there is a difference in the ratings if the model shown in the advertisement has blue eyes rather than brown eyes. \\

\question{
    6.1 a
}{
    Read in the file \dataset{eyeColor.csv} and store the data in an object called \rcode{dataset8}.
}

\rcodeanswertiny{6.1a}

Note that this data set contains all four groups (\rcode{Blue}, \rcode{Brown}, \rcode{Green}, \rcode{Down}). Run the following \texttt{R} code to isolate the scores of the groups that were shown advertisements with \rcode{Blue} and \rcode{Brown} eyes and store them in the new \rcode{ttestData} object. \\

\codeblock{ttestData  <- subset(dataset8,\\ 
                     \hspace*{110pt}dataset8\$Group == \textquotesingle Blue\textquotesingle | \\
                     \hspace*{110pt}dataset8\$Group == \textquotesingle Brown\textquotesingle)}

\question{
    6.1 b
}{
    Write down the \concept{null hypothesis} $H_0$ and \concept{alternative hypothesis} $H_1$ for testing whether the \concept{mean} score of the group that was shown \rcode{Blue} eyes is equal to the \concept{mean} score of the group that was shown \rcode{Brown} eyes. Remember that these are independent \concept{samples}.
}

\hypothesesbox{6.1b}

\question{
    6.1 c
}{
    Use the \rcode{t.test()} function to test the equality of the two means.
}

\hint{Hint 6.1: Make sure to specify \rcode{var.equal = TRUE} to perform a \concept{two-sample t-test} instead of the non-parametric \concept{Welch’s t-test}.}

\rcodeanswertiny{6.1c}

\question{
    6.1 d
}{
    What is your conclusion on the basis of these results? Include the following elements:
        \begin{itemize}
        \item[$\square$] Discuss what the \concept{p-value} is for this test.
    \item[$\square$] Discuss whether $H_0$ is rejected or not.
    \item[$\square$] Describe what this tells us about $\mu_{Blue}$ and $\mu_{Brown}$.  
    \item[$\square$] Describe what type of error is relevant \textit{(type-I or type-II)}.
\end{itemize}
}

\sixlineanswerbox{6.1d}

Now, instead of using the \rcode{t.test()} function to test the equality of the two \concept{means}, you can test the equality of the two \concept{means} using a \concept{regression model}. Therefore, you need to add a variable to our data that says whether a participant saw one of the two groups (e.g., only the people that were shown \rcode{Brown} eyed models). \\

Run the following code in R: \\

\codeblock{dummyBrown <- as.numeric(ttestData\$Group == \textquotesingle Brown\textquotesingle)\\
ttestData <- cbind(ttestData, dummyBrown)}

\clearpage % Page break

\question{
    6.1 e
}{
    What are the contents of \rcode{dummyBrown}? What do we call this kind of variable?
}

\twolineanswerbox{6.1e}

\question{
    6.1 f
}{
    Create a \concept{linear model} in \texttt{R} where you predict the (outcome) variable \rcode{Score} using only the (predictor) variable \rcode{dummyBrown}. Store the fitted model in an object called \rcode{ttestreg}.
}

\rcodeanswersmall{6.1f}

\question{
    6.1 g
}{
    Use the \rcode{summary()} function to inspect the results of the \rcode{ttestreg} model. How does this output correspond to the \concept{t-test} that you performed in assignment 6.1c? Where do you find the \concept{p-value} that you calculated using the \rcode{t.test()} function?
}

\rcodeanswersmall{6.1g}

\twolineanswerbox{6.1g}

\clearpage % Page break