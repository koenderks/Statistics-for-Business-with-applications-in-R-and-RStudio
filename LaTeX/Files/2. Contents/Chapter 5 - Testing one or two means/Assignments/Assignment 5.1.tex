\setcounter{chapter}{5}
\setcounter{section}{1}
\setcounter{question}{0}
\setcounter{hint}{0} % Only for first assignment in chapter

%%%%%%%%%%%%%%%%%%%%%%%%%%%%%%%%%%%%%%%%%%%%%%%%%%%%%%%%%%%%%%%%%%%%%%%%%%%
% Assignment 5.1: Hypothesis testing by hand using the z- and t-score
%%%%%%%%%%%%%%%%%%%%%%%%%%%%%%%%%%%%%%%%%%%%%%%%%%%%%%%%%%%%%%%%%%%%%%%%%%%

\handassignment{Hypothesis testing by hand using the z- and t-score}

The maximum allowed amount of PFAS chemicals in soil is 0.9 microgram per kg of dry soil. In a regular soil check, researchers take 49 soil \concept{samples} in a town near a plastic factory and find a \concept{mean} PFAS level of 0.918 microgram/kg with a \concept{standard deviation} of 0.071. They want to show with 95\% confidence that the PFAS levels in the town are significantly above the norm. \\

\question{
    Write down the relevant information from this case.
}

\twolineanswerbox

\question{
    Write down the \concept{null hypothesis} $H_0$ and the \concept{alternative hypothesis} $H_1$ for the researchers’ test of the \concept{mean} PFAS level against the norm.
}

\hypothesesbox

The \concept{critical z-value} for a one-sided 95\% confidence interval is equal to 1.645 (see also Table 2 on page~\pageref{table2}). \\

\clearpage % Page break

\question{
    Calculate the \concept{lower bound} of the one-sided \concept{confidence interval} for the PFAS level \concept{population mean} with 95\% confidence.
}

\emptyanswerbox{
    Lower bound: \shortanswerline
    \answerskip
    Calculation:
    \answerskip
    \rule{\textwidth}{0.4pt}
}

\question{
    Draw the conclusion for the researchers. Include the following four elements:
        \begin{itemize}
    \item[$\square$] Show how $\mu_0$ relates to the \concept{confidence interval.}
    \item[$\square$] Discuss whether $H_0$ is rejected or not.
    \item[$\square$] Describe what this tells us about $\mu$ and $\mu_0$.
    \item[$\square$] Describe what type of error is relevant \textit{(type-I or type-II)}.
\end{itemize}
}

\sixlineanswerbox

\question{
    Calculate the \concept{z-score} for this situation.
}

\hint{You can find the formula for the \concept{z-score} in the formula sheet on page~\pageref{formulasheet}. Use $\mu_0$ in this formula.}

\emptyanswerbox{
    z-score: \shortanswerline
    \answerskip
    Calculation:
    \answerskip
    \rule{\textwidth}{0.4pt}
}

\clearpage % Page break

\question{
    Compare the calculated \concept{z-score} with the \concept{critical z-value}. What does this tell you about the \concept{p-value}?
}

\hint{Use the absolute value of the calculated \concept{z-score}.}

\twolineanswerbox

\question{
    Draw the conclusion again, but now using the \concept{z-score}, is it the same? Include the following elements:
    \begin{itemize}
    \item[$\square$] Show how the calculated \concept{z-score} relates to the \concept{critical z-score}.
    \item[$\square$] Discuss whether $H_0$ is rejected or not.
    \item[$\square$] Describe what this tells us about $\mu$ and $\mu_0$.
    \item[$\square$] Describe what type of error is relevant \textit{(type-I or type-II)}.
\end{itemize}
}

\sixlineanswerbox

The researchers want to quickly check another town near the plastic factory, but have less time available. They take 16 \concept{samples} in the second town and find a much higher \concept{mean} PFAS level of 0.930 microgram/kg with a comparable \concept{standard deviation} of 0.070. \\

\question{
    Calculate the \concept{t-score} for this situation.
}

\hint{You can find the formula for the \concept{t-score} in the formula sheet on page~\pageref{formulasheet}. Use $\mu_0$ in this formula.}

\clearpage % Page break

\emptyanswerbox{
    t-score: \shortanswerline
    \answerskip
    Calculation:
    \answerskip
    \rule{\textwidth}{0.4pt}
}

The \concept{critical t-value} for a one-sided 95\% \concept{confidence interval} with 15 \concept{degrees of freedom} is equal to 1.753.

\question{
    Draw the conclusion for the researchers using the (absolute) t-score. Include the following elements:
    \begin{itemize}
        \item[$\square$] Show how the calculated \concept{t-score} relates to the \concept{critical t-score}.
    \item[$\square$] Discuss whether $H_0$ is rejected or not.
    \item[$\square$] Describe what this tells us about $\mu$ and $\mu_0$.
    \item[$\square$] Describe what type of error is relevant \textit{(type-I or type-II)}.
\end{itemize}
}

\sixlineanswerbox

\question{
    Also calculate the \concept{lower bound} of the one-sided \concept{confidence interval} for the PFAS level \concept{population mean} with 95\% confidence.
}

\emptyanswerbox{
    Lower bound: \shortanswerline
    \answerskip
    Calculation:
    \answerskip
    \rule{\textwidth}{0.4pt}
}

\clearpage % Page break

\question{
    If you would draw the conclusion based on this \concept{lower bound} would you get the same result? Explain why.
}

\emptyanswerbox{
    The conclusion \textbf{would} / \textbf{would not} be the same.
    \answerskip
    Explanation:
    \answerskip
    \rule{\textwidth}{0.4pt}
    \answerbreak
    \rule{\textwidth}{0.4pt}
}

\question{
    Why is it that even though the second town shows much higher PFAS level in the \concept{sample} (0.930 vs 0.918) $H_0$ cannot be rejected?
}

\twolineanswerbox

\clearpage % Page break