%%%%%%%%%%%%%%%%%%%%%%%%%%%%%%%%%%%%%%%%%%%%%%%%%%%%%%%%%%%%%%%%%%%%%%%%%%%
% Assignment 3.1: Confidence interval and hypothesis testing by hand
%%%%%%%%%%%%%%%%%%%%%%%%%%%%%%%%%%%%%%%%%%%%%%%%%%%%%%%%%%%%%%%%%%%%%%%%%%%

\handassignment{Assignment 3.1: Confidence interval and hypothesis testing by hand}

A call center with 38 employees handles thousands of calls a day. The company wants to get more insights into the duration of calls and the workload of the employees. On a day with a total of 4513 calls, the company decides to randomly pick 100 calls and measure how long they take. The \concept{mean} call duration for this sample is 145 seconds with a \concept{standard deviation} of 25 seconds. The call center wants to use this information to estimate the \concept{mean} call duration for that day. \\

\question{
    3.1 a
}{
    Write down the relevant information from this case. Use the symbols $N$, $n$, $\bar{x}$, $s$, $\sigma$, and $\mu$.
}

\hint{Hint 3.1: Not all symbols are known and some should be left empty.}

\emptyanswerbox{
    3.1a
}{
        $N$: \shortanswerline \hspace{1cm} \quad $s$: \shortanswerline
            \answerbreak
            $n$: \shortanswerline \hspace{1cm} \quad $\sigma$: \shortanswerline
            \answerbreak
            $\bar{x}$: \shortanswerline \hspace{1cm} \quad $\mu$: \shortanswerline
}

\question{
    3.1 b
}{
    What is the best estimate of the \concept{mean} call duration ($\mu$) based on this sample?
}

\emptyanswerbox{
    3.1b
}{
    $\mu = $\shortanswerline    
    \answerskip
    Explanation:
    \answerskip
    \rule{\textwidth}{0.4pt}
}

\clearpage % Page break

\question{
    3.1 c
}{
    Calculate the \concept{standard error} of the mean ($SE_\mu$) based on this sample.
}

\hint{Hint 3.2: Check the formula sheet to find out how to calculate $SE_\mu$.}

\emptyanswerbox{
    3.1c
}{
    $SE_\mu = $\shortanswerline    
    \answerskip
    Calculation:
    \answerskip
    \rule{\textwidth}{0.4pt}
}

\question{
    3.1 d
}{
    Calculate the 99\% \concept{confidence interval} for the mean call duration.
}

\hint{Hint 3.3: You can find the formula for the \concept{lower bound}, \concept{upper bound} and \concept{z-value} in the formula sheet.}

\emptyanswerbox{
    3.1d
}{
    z-value: \qquad \shortanswerline    
    \answerskip
    Calculation: \mediumanswerline
    \answerskip
    \answerskip
    Lower bound: \shortanswerline    
    \answerskip
    Calculation: \mediumanswerline
    \answerskip
    \answerskip
    Upper bound: \shortanswerline    
    \answerskip
    Calculation: \mediumanswerline
    \answerskip
}

The manager of the call center does not really care about the \concept{lower bound} and he does not require such high confidence. He just wants to make sure the average workload per employee is not too high, because otherwise he is forced by employment laws to hire more people. He calculated that if the \concept{mean} call duration is below 150 seconds the workload is acceptable, and wants to use this sample to show with 95\% confidence that the \concept{mean} call duration is below 150 seconds. \\

\clearpage % Page break

\question{
    3.1 e
}{
    Write down the hypotheses $H_0$ and $H_1$ for the manager’s test. What is the value of $\mu_0$? Also describe $\mu_0$ for this case. 
}

\hint{Hint 3.4: This is a one-sided test. Consider this when you formulate the hypotheses.}

\emptyanswerbox{
    3.1e
}{
    $\mu_0$: \shortanswerline    
    \answerskip
    $H_0$: \shortanswerline \hspace{5cm} $H_1$: \shortanswerline
}

\question{
    3.1 f
}{
    Do you need the \concept{lower bound} or the \concept{upper bound} of the \concept{confidence interval} for this test? Calculate this bound. 
}

\emptyanswerbox{
    3.1f
}{
    ........ bound: \shortanswerline    
    \answerskip
    Calculation: \qquad \hspace*{5pt} \mediumanswerline
}

\question{
    3.1 g
}{
    Draw the conclusion for the manager. Include the following four elements:
    \begin{itemize}
        \item[$\square$] Show how the \concept{confidence interval} relates to $\mu_0$.
        \item[$\square$] Discuss whether $H_0$ is rejected or not.
        \item[$\square$] Describe what this tells us about $\mu$ and $\mu_0$.
        \item[$\square$] Describe what type of error is relevant \textit{(type-I or type-II)}.
    \end{itemize}
}

\sixlineanswerbox{3.1g}

\clearpage % Page break