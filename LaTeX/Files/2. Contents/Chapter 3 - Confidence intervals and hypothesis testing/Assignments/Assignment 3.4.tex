\setcounter{section}{3}
\setcounter{subsection}{4}
\setcounter{question}{0}

%%%%%%%%%%%%%%%%%%%%%%%%%%%%%%%%%%%%%%%%%%%%%%%%%%%%%%%%%%%%%%%%%%%%%%%%%%%
% Assignment 3.4: Inspect and test the normality of a sample in R
%%%%%%%%%%%%%%%%%%%%%%%%%%%%%%%%%%%%%%%%%%%%%%%%%%%%%%%%%%%%%%%%%%%%%%%%%%%

\rassignment{Inspecting and testing the normality of a sample in R}

In this assignment you are going to work with an \texttt{R} package called \rcode{car}. Packages are extensions for \texttt{R} created by its community. They contain functions that are not available in the basic \texttt{R} version and can be really useful. There are many packages on the internet, so only use the one you really need and always make sure you use packages from a reliable source (like the official \rcode{CRAN} servers). For clarity, it will be indicated what functions come from what packages in the chapters. If there is no package mentioned, functions come from base \texttt{R}. \\

A package has to be installed once. Run the following code in R to install the car package: \\

\codeblock{install.packages(\textquotesingle car\textquotesingle)}

A package needs to be loaded in \textbf{every script} that uses functions from this package. \\

\codeblock{library(car)}

This assignment assumes you have done assignment 3.3 and therefore have four $n = 90$ samples called \rcode{sample1} - \rcode{sample4} from populations \rcode{P1} - \rcode{P4} from \rcode{dataset5} and calculated the \concept{confidence intervals} for the population means. \\

\question{
    Create a histogram for each sample in \rcode{sample1} - \rcode{sample4}.
}

\rcodeanswermedium

\question{ 
    Which samples look like they could have been taken from a \concept{normal distribution}?
}

\threelineanswerbox

\clearpage % Page break

\question{
    Use this \texttt{R} code (from the \rcode{car} package) to create four \concept{qq-plots} for the four \concept{samples}. 
}

\codeblock{qqPlot(sample1, distribution = \textquotesingle norm\textquotesingle) {\color{dataset}\# qqPlot for sample 1}}

\rcodeanswermedium

\question{
    Evaluate each \concept{qq-plot} and explain the deviations from the diagonal by referencing features of the histograms. Do the \concept{qq-plots} confirm your answer for question 3.4b?
}

\fourlineanswerbox

\question{
    Formulate the \concept{null hypothesis} $H_0$  and \concept{alternative hypothesis} $H_1$ for a test of normality in these samples.
}

\hypothesesbox

\question{
    Use the following \texttt{R} code to perform a \concept{Shapiro-Wilk} normality test on the four samples. Use a confidence level of 99\% and write down the conclusion for each sample.
}

\codeblock{shapiro.test(sample1) {\color{dataset}\# Normality test for sample 1}}

\hint{First think about what \concept{p-values} lead to rejecting the \concept{null hypothesis} $H_0$.}

\clearpage % Page break

\rcodeanswersmall

\fourlineanswerbox

In assignment 3.3 you used these \concept{samples} and the \concept{normal distribution} to estimate the \concept{confidence interval} for the \concept{population mean}. You also tested whether the actual population means were inside these intervals, which they almost always were. \\

Now you found out that most of these samples are actually not normally distributed at all. \\

\question{
    If a sample is not normally distributed, does that mean you cannot use it to estimate the population mean? Explain your answer.
}

\emptyanswerbox{
    When the sample is not normally distributed, the sample \textbf{can} / \textbf{cannot} be used to estimate the population mean.
    \answerskip
    Explanation:
    \answerskip
    \rule{\textwidth}{0.4pt}
    \answerbreak
    \rule{\textwidth}{0.4pt}
    \answerbreak
    \rule{\textwidth}{0.4pt}
}

\clearpage % Page break