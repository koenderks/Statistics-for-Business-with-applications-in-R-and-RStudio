\setcounter{chapter}{1}
\setcounter{section}{2}
\setcounter{question}{0}

%%%%%%%%%%%%%%%%%%%%%%%%%%%%%%%%%%%%%%%%%%%%%%%%%%%%%%%%%%%%%%%%%%%%%%%%%%%
% Assignment 1.2: Descriptive statistics of small data sets in R
%%%%%%%%%%%%%%%%%%%%%%%%%%%%%%%%%%%%%%%%%%%%%%%%%%%%%%%%%%%%%%%%%%%%%%%%%%%

\rassignment{Descriptive statistics of small data sets in R}

This assignment assumes you opened a new script in the code editor in
\texttt{RStudio}. Write and run your own code to answer the following questions. \\

In R, a one-dimensional row of numbers is represented by a \rcode{vector}. \\

\question{
    Use the \rcode{c()} function to enter the numbers from assignment 1.1a in a new vector called \rcode{dataset1}. Next, run the following code in \texttt{R} and explain what you see.
}

\codeblock{View(dataset1)}

\hint{Check Part I of the R help on page~\pageref{rhelp} for more information on how to make a \rcode{vector}.}

\rcodeanswersmall
\twolineanswerbox

\question{
    Write and run your own code in R to find the \concept{mean}, \concept{mode}, \concept{median}, and \concept{range} for the vector \rcode{dataset1}. Compare your answers with those of assignment 1.1a.
}

\hint{Check part IV of the R help on page~\pageref{rhelppartfour} for descriptive statistics functions.}

\hint{There is no \concept{mode} function in R, but you can find the mode in a frequency \rcode{table}.}

\rcodeanswersmall

\clearpage % Page break

\emptyanswerbox{  
    Mean: \quad \shortanswerline \hspace{.5cm} \quad Median: \quad \quad \shortanswerline
                \answerskip
                Mode: \quad \shortanswerline \hspace{.5cm} \quad Range: \qquad \quad \shortanswerline
}

\question{
    Run the following code in \texttt{R} and explain what you see.
}

\codeblock{quantile(dataset1, type = 6)}

\twolineanswerbox

\question{
    Use the \rcode{c()} function to create a vector \rcode{dataset2} with the data from assignment 1.1b and find the \concept{mean}, \concept{mode}, \concept{median}, \concept{range}, and \concept{quartiles} for these data. Compare your answers with the answers of assignment 1.1b.
}

\rcodeanswersmall

\emptyanswerbox{  
Mean: \quad \shortanswerline \hspace{.5cm} \quad Range: \qquad \qquad \qquad \qquad \shortanswerline
            \answerbreak
            Mode: \quad \shortanswerline \hspace{.4cm} \quad Lower quartile: \qquad \quad \shortanswerline
            \answerbreak
            Median: \shortanswerline \hspace{.4cm} \quad Upper quartile: \qquad \quad \shortanswerline
}
            
\clearpage % Page break