\subsection{Data types of objects}

\beginnerexercise{
    Run the following code in \texttt{R}: \\
    \\
    \codeblock{a1 <- \textquotesingle 1\textquotesingle; a2 <- 1; a3 <- TRUE}
    What are the data types of \rcode{a1}, \rcode{a2}, and \rcode{a3}? Now run the following code in \texttt{R}: \\
    \\
    \codeblock{b1 <- c(a1, a2); b2 <- c(a1, a3); b3 <- c(a2, a3); b4 <- c(a1, a2, a3)}
    What are the modes of \rcode{b1}, \rcode{b2}, \rcode{b3}, and \rcode{b4}? Can you explain?
}

\beginnerexercise{
    Why does \rcode{TRUE/TRUE} yield 1, \rcode{FALSE/TRUE} yield 0, \rcode{FALSE/FALSE} yield \rcode{NaN}, and \rcode{TRUE/FALSE} yields \rcode{Inf}?
}

\beginnerexercise{
    Convert the logical \rcode{TRUE} to a numerical. Convert this numerical to a character. Next, convert the logical \rcode{TRUE} to a character. Why do these results differ? (Hint: look at as.numeric())
}

\beginnerexercise{
    How can you check whether a vector is numeric? Is the vector \rcode{c(1,0)} a numeric vector? And the vector \rcode{c(TRUE, FALSE)}? And the vector \rcode{c(TRUE, FALSE, 1, 0)}? Why? (Hint: look at is.numeric())
}

\clearpage