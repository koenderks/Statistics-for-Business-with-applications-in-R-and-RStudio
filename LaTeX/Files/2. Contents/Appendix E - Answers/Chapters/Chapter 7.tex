\setcounter{chapter}{7}
\setcounter{section}{1}
\setcounter{answer}{0}

\section{Chapter 7: Comparing proportions and distributions}

\answer{
    $H_0$: The 2019 distribution is equal to the historical distribution \\
    $H_1$: The 2019 distribution is not equal to the historical distribution
}

\answer{
Expected number = Historical \% $\times$ Observed number
    \begin{center}
    \begin{tabular}{|r|c|c|c|c|c|}
    \cline{2-6} 
    \multicolumn{1}{c|}{} & Historical & Observed ($O$) & Expected ($E$) & $O - E$ & $\frac{(O - E)^2}{E}$ \tstrut\bstrut\\
    \hline
    Spring & 4.87 (30\%) & 2.90 & 25.2 & 11.8 & 5.525 \tstrut\bstrut\\
    \hline
    Summer & 3.04 (40\%) & 4.50 & 33.6 & -6.6 & 1.296 \tstrut\bstrut\\
    \hline
    Fall & 1.65 (15\%) & 4.94 & 12.6 & -0.6 & 0.029 \tstrut\bstrut\\
    \hline
    Winter & 2.88 (15\%) & 3.28 & 12.6 & -4.6 & 1.679 \tstrut\bstrut\\
    \noalign{\hrule height 2pt}
    Total & 2.31 (100\%) & 4.73 & 84 &  & 8.530 \tstrut\bstrut\\
    \cline{1-4} \cline{6-6}
    \end{tabular}
\end{center}
}

\answer{
    Since every season has an expected value above 5 you can do a chi-square test.
}

\answer{
    $X^2 = 8.530$
}

\answer{
    The calculated chi-square value of 8.530 is higher than the critical chi-square value of 7.8. $H_0$ is rejected. The 2019 distribution is significantly different from the historical distribution. There is a 5\% risk of a type-I error.
}

\answer{
    You rejected the null hypothesis $H_0$ with 95\% confidence, and so the p-value must be lower than 0.05.
}

\answer{
    $X^2 = 8.530$ \\
    \\
    \answercode{Observed <- c(37, 27, 12, 8) \\
Historical <- c(0.3, 0.4, 0.15, 0.15) \\
Expected <- c(25.2, 33.6, 12.6, 12.6) \\
 \\
chi <- sum((Observed - Expected)\textrm{\textasciicircum}2 / Expected) {\color{dataset}\# 8.530}}
}

\answer{
    Yes. \\
    \\
    \answercode{qchisq(p = 0.95, df = 3) {\color{dataset}\# 7.185}}
}

\clearpage % Page break

\answer{
    \texttt{R} returns the same chi-squared as calculated, so the answer was correct. It also shows a p-value of below 0.05, as expected. \\
    \\
    \answercode{{\color{dataset}\# chi-square test: x = observations p = model distribution} \\
{\color{dataset}\# rescale makes sure the model distribution adds up to 100\%} \\
chisq.test(x = Observed, p = Historical, rescale.p = TRUE) \\
 \\
{\color{dataset}\# Chi-squared value: 8.5298} \\
{\color{dataset}\# p-value: 0.0362}}
}

\answer{
    The expected values are 25.2, 33.6, 12.6, and 12.6. \texttt{R} shows the same expected values.\\
    \\
    \answercode{chisq <- chisq.test(x = Observed, p = Historical) \\
chisq\$expected   {\color{dataset}\# Extract expected values with \$expected} \\
{\color{dataset}\# 25.2 33.6 12.6 12.6}}
}

\answerbreakline \setcounter{section}{2} \setcounter{answer}{0}

\answer{
    The \rcode{sales} data frame contains 3 columns: \rcode{month}, \rcode{historical} and \rcode{newstore}. It contains the ‘Historical’ and ‘New Store’ distribution of sales over the months. \\
    \\
    \answercode{{\color{dataset}\# These are the values for the sales data set} \\
sales <- data.frame(month = seq(from = 1, to = 12, by = 1), \\
                    \hspace*{110pt}historical = c(5.1, 5.1, 6.7, 10, 11.4, 10, \\
                                   \hspace*{192pt}6.7, 5.1, 6.7, 10, 11.7, 11.7),   \\
                    \hspace*{110pt}newstore = c(5.6, 6.2, 9.4, 8.6, 6.8, 4.8, \\
                                 \hspace*{180pt}5.6, 4.8, 8.8, 12.6, 13.1, 13.7)) \\
\\
summary(sales)}
}

\answer{
    The chi-square test requires every cell in the expected distribution to have a value of at least 5 and it requires the total of both groups to be equal. Since the historical distribution contains more than 5 in every cell and both observed and expected values add up to the same number (100) we can use this for a chi-squared test.
}

\answer{
    $H_0$: The new distribution is equal to the historical distribution \\
    $H_1$: The new distribution is not equal to the historical distribution
}

\clearpage % Page break

\answer{
    The p-value of 0.6963 is higher than the critical p-value of 0.10. $H_0$ is not rejected. The new store distribution is not significantly different from the historical distribution. There is a risk a type-II error. \\
    \\
    \answercode{chisq.test(x = sales\$newstore, p = sales\$historical, rescale.p = TRUE) \\
{\color{dataset}\# p-value: 0.6963}}
}

\answerbreakline \setcounter{section}{3} \setcounter{answer}{0}

\answer{
    The best estimate of the population proportion $\pi$ is the sample proportion $p$. \\
    \\
    $\pi_1 = \frac{k}{n} = \frac{8}{71} = 0.113$ \\
    \\
    $\pi_2 = \frac{k}{n} = \frac{16}{111} = 0.144$
}

\answer{
    Confidence interval sample 1: \\
    \\
    $p \pm z_\alpha \times \sqrt{\frac{p(1-p)}{n}} = 0.113 \pm 1.960 \times \sqrt{\frac{0.113 \times (1 - 0.113)}{71}} = [ 0.039;\, 0.187 ]$ \\
    \\
    Confidence interval sample 2: \\
    \\
    $p \pm z_\alpha \times \sqrt{\frac{p(1-p)}{n}} = 0.144 \pm 1.960 \times \sqrt{\frac{0.144 \times (1 - 0.144)}{111}} = [ 0.078;\, 0.210 ]$  
}

\answer{
    $H_0$: $\pi_2 \leq \pi_1$ \hspace*{4cm} $H_1$: $\pi_2 > \pi_1$ \\
    \\
    Where $\pi_2$ and $\pi_1$ are the success proportions for the evening and afternoon calls respectively.
}

\answer{
    Combined success probability: \\
    \\
    $p^* = \frac{k_1 + k_2}{n_1 + n_2} = \frac{8 + 16}{71 + 111} = 0.132$
}

\answer{
    Combined standard error: \\
    \\
    $s_p = \sqrt{p^*(1-p^*)(\frac{1}{n_1} + \frac{1}{n_2})} = \sqrt{0.132(1-0.132)(\frac{1}{71} + \frac{1}{111})} = 0.0514$
}

\answer{
    z-score: $\frac{p_1 - p_2}{s_p} = \frac{0.144 - 0.113}{0.514} = 0.612$ \\
    \\
    Note that $p_1$ and $p_2$ were switched because in the hypotheses $\pi_1$ and $\pi_2$ were also switched.
}

\answer{
    The calculated z-score of 0.612 is lower than  than the critical z-value of 1.645. $H_0$ is not rejected. The evening success rate is not shown to be significantly higher than the afternoon success rate. There is a risk of a type-II error.
}

\clearpage % Page break

\answer{
    The code creates a vector of successes \rcode{k} and a vector of sample sizes \rcode{n}. The proportion test then tests the equality. It shows the proportions you calculated earlier, and a  p-value of 0.6984 which supports your conclusion if you do not reject $H_0$. \\
    \\
    \answercode{n <- c(71, 111) \\
k <- c(8, 16) \\
prop.test(x = k, n = n) \\
{\color{dataset}\# p-value: 0.6984}
} \\
\\
Note that \texttt{R} actually returns a chi-squared value. Because it actually does a chi-square test it can in fact be used to test more than two proportions at the same time.
}

\clearpage % Page break