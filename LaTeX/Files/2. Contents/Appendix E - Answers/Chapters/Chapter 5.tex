\setcounter{chapter}{5}
\setcounter{section}{1}
\setcounter{answer}{0}

\section{Chapter 5: Comparing one or two means}

\answer{
    $n = 49$ \\
    $\bar{x} = 0.918$ \\
    $s = 0.071$ \\
    $\mu_0 = 0.9$
}

\answer{
    $H_0$: $\mu_0 \leq 0.9$ \hspace{4cm} $H_1$: $\mu_0 > 0.9$
}

\answer{
    Lower bound: $\bar{x} - z_{\alpha} \times SE_\mu = 0.918 - 1.645 \times \frac{0.071}{\sqrt{49}} = 0.901$
}

\answer{
    The lower bound of the confidence interval for $\mu$ is higher than $\mu_0$. $H_0$ is rejected with 95\% confidence. The PFAS level in town is significantly higher than 0.9 microgram/kg dry soil. There is a risk of 5\% for a type-I error.
}

\answer{
    z-score: $\frac{\bar{x} - \mu}{\nicefrac{s}{\sqrt{n}}} = \frac{0.918 - 0.9}{\nicefrac{0.071}{\sqrt{49}}} = 1.775$
}

\answer{
    The critical z-value for 95\% confidence is 1.645. The p-value (the probability that $H_0$ is true) is 0.05. For any z-value higher than 1.645 the p-value is lower than 0.05. Since for this test the (absolute of the) calculated z-score  = 1.775 and this is higher than 1.645, this means that the p-value is lower than 0.05. \\
\\
Note:  For right sided tests the z-score is negative, but since the standard normal distribution is symmetric we can simply use the positive value for comparing it with the critical z-value.
}

\answer{
    The calculated z-score is more extreme (higher) than the critical z-value of 1.645. $H_0$ is rejected with 95\% confidence. The PFAS level in town is significantly higher than 0.9 microgram/kg dry soil. There is a risk of 5\% for a type-I error. The two methods produce the same answer. It cannot be different; the methods are equivalent.
}

\answer{
    t-score: $\frac{\bar{x} - \mu}{\nicefrac{s}{\sqrt{n}}} = \frac{0.930 - 0.9}{\nicefrac{0.070}{\sqrt{16}}} = 1.714$
}

\answer{
  The calculated t-score of 1.714 is lower than the critical t-value of 1.753. $H_0$ is not rejected. The PFAS level in town is significantly below the norm. There is a risk type-II error.  
}

\answer{
    Lower bound: $\bar{x} - t_{\alpha (df = 15)} \times SE_\mu = 0.930 - 1.753 \times \frac{0.070}{\sqrt{16}} = 0.899$
}

\answer{
    Yes. The lower bound for the population mean $\mu$ is 0.899, it can therefore not be ruled out that the mean PFAS level is below 0.9.
}

\answer{
    The sample is so much smaller (16 vs. 49) that there is too much uncertainty in the result.
}

\answerbreakline \setcounter{section}{2} \setcounter{answer}{0}

\clearpage % Page break

\answer{
    Result: 1.645 \\
    \\
    This is the z-value for a 95\% one-sided confidence interval. \\
    \\
    \answercode{qnorm(p = 0.95, mean = 0, sd = 1, lower.tail = TRUE) \\
{\color{dataset}\# Or because the standard normal distribution is the default simply use:}  \\
qnorm(0.95) {\color{dataset}\# 1.645}}
}

\answer{
    Because for a two-sided interval you spread the risk over the two tails. There is 2.5\% of risk on the left tail and 2.5\% of risk on the right tail. \\
    \\
    \answercode{qnorm(p = 0.975) {\color{dataset}\# 1.960}}
}

\answer{
    The \rcode{pnorm()} is the inverse of the \rcode{qnorm()} function: It returns the cumulative probability for a given value \rcode{q} in a specified normal distribution. \\
    \\
    \answercode{pnorm(q = 1.645, mean = 0, sd = 1) {\color{dataset}\# 0.95} \\
pnorm(1.960) {\color{dataset}\# 0.975}}
}

\answer{
    The \rcode{qt()} function returns the t-value for a given cumulative probability \rcode{p} and given number of degrees of freedom \rcode{df}. \\
    \\
    \answercode{qt(p = 0.95, df = 15) {\color{dataset}\# 1.753}}
}

\answer{
    The \rcode{pt()} is the inverse of the \rcode{qt()} function: It returns the cumulative probability for a given value \rcode{q} in a specified t-distribution. \\
    \\
    \answercode{pt(q = 1.753, df = 15) {\color{dataset}\# 0.95}}
}

\answer{
\vspace*{-7.5pt}
    \begin{minipage}{.5\textwidth}
    t$_1$: -1.06 \\
    t$_3$: 1.328
    \end{minipage}
        \begin{minipage}{.5\textwidth}
            t$_2$: -1.328 \\
    t$_4$: 1.328
    \end{minipage} \\
    \\
    \\
    \answercode{qt(p = 0.15, df = 19)                       {\color{dataset}\# -1.06} \\
qt(p = 0.10, df = 19)                       {\color{dataset}\# -1.328} \\
qt(p = 0.90, df = 19)                       {\color{dataset}\# 1.328} \\
qt(p = 0.10, df = 19, lower.tail = FALSE)   {\color{dataset}\# 1.328}}
}

\clearpage % Page break

\answer{
    \begin{minipage}{.3\textwidth}
    p$_1$: 0.081 
    \end{minipage}
        \begin{minipage}{.3\textwidth}
            p$_2$: 0.929 
    \end{minipage}
            \begin{minipage}{.3\textwidth}
            p$_2$: 0.015
    \end{minipage} \\
    \\
    \answercode{pt(q = -1.5, df = 11)                     {\color{dataset}\# 0.081} \\
diff(pt(q = c(-2,2), df = 11))            {\color{dataset}\# 0.929} \\
pt(q = 2.5, df = 11, lower.tail = FALSE)  {\color{dataset}\# 0.015}}
}

\answer{
    Two tailed inequality test: 2.467 \\
One-tailed right sided test: 2.153 \\
One-tailed left sided test: -2.153 \\
\\
\answercode{{\color{dataset}\# Two-tailed inequality test} \\
qt(p = 0.99, df = 28)  {\color{dataset}\# 2.467}   \\
 \\
{\color{dataset}\# One-tailed right sided test}      \\             
qt(p = 0.98, df = 28)  {\color{dataset}\# 2.153} \\
 \\
{\color{dataset}\# One-tailed left sided test} \\
qt(p = 0.98, df = 28, lower.tail = FALSE) {\color{dataset}\# -2.153}}
}

\answer{
    Two-tailed inequality test: $H_0$ rejected (2.6 > 2.476) \\
One-tailed right sided test: $H_0$ rejected (-2.3 < -2.153) \\
One-tailed left sided test: $H_0$ not rejected (1.6 < 2.153) \\

}

\answerbreakline \setcounter{section}{3} \setcounter{answer}{0}

\answer{
    Because different men get the caffeine and the placebo. There are 18 unique test subjects. Everyone gets tested once and every measurement is therefore independent.
}

\answer{
    $H_0$: $\mu_1 \geq \mu_2$ \hspace{4cm} $H_1$: $\mu_1 < \mu_2$
}

\answer{
    $s^2_p = \frac{(n_1 - 1) s_1^2 + (n_2 - 1) s_2^2}{n_1 + n_2 - 2} = \frac{(9 - 1) \times 6.49^2 + (9 - 1) \times 8.14^2}{9 + 9 - 2} = 54.15$
}

\answer{
    $t = \frac{(x_1 - x_2) - D_0}{\sqrt{s^2_p (\frac{1}{n_1} + \frac{1}{n_2})}} = \frac{(94.11 - 101.22)}{\sqrt{54.15 \times (\frac{1}{9} + \frac{1}{9})}} = -2.050$
}

\answer{
    The calculated t-score of -2.050 is more extreme (lower) than the critical value of -1.746. $H_0$ is rejected with 95\% confidence. The mean caffeine RER level is significantly lower than the mean placebo RER level. There is a risk of 5\% for type-I error.
}

\clearpage % Page break

\answer{
    \vspace*{-7.5pt}
    \begin{minipage}[t]{0.5\textwidth}
    \rcode{placebo}	\\
Mean: 	101.22 \\
Standard deviation: 8.14
    \end{minipage}
    \begin{minipage}[t]{0.5\textwidth}
    \rcode{caffeine} \\
    Mean: 94.11 \\
    Standard deviation: 6.49
    \end{minipage} \\
    \\
    \\
    \answercode{{\color{dataset}\# These are the values for the RER test data set} \\
placebo <- c(97, 106, 120, 104, 96, 100, 93, 96, 99) \\
caffeine <- c(97, 92, 95, 100, 95, 88, 85, 106, 89) \\
 \\
 \\
mean(placebo)  {\color{dataset}\# Mean:                 101.22} \\
sd(placebo)    {\color{dataset}\# Standard deviation:   8.14}   \\
 \\
mean(caffeine) {\color{dataset}\# Mean:                 94.11}\\
sd(caffeine)   {\color{dataset}\# Standard deviation:  6.49} }
}

\answer{
    The code runs an independent (not paired) samples t-test with a confidence of 95\% for \rcode{placebo} and \rcode{caffeine}, with the alternative hypothesis $H_1$ that the mean in the caffeine group is lower than the mean in the placebo group. The variances are assumed equal. The resulting t-value and p-value confirm the previous result to reject $H_0$. \\
    \\
    \answercode{t.test(x = caffeine, y = placebo, alternative = \textquotesingle less\textquotesingle, mu = 0, \\
\hspace*{40pt}paired = FALSE, var.equal = TRUE, conf.level = 0.95)
}
}

\answer{
    The Welch Two Sample t-test leads to a slightly different p-value of 0.02899, but the same conclusion: rejection of $H_0$. \\
    \\
    \answercode{t.test(x = caffeine, y = placebo, alternative = \textquotesingle less\textquotesingle, mu = 0, \\
\hspace*{40pt}paired = FALSE, var.equal = FALSE, conf.level = 0.95)
}
}

\answer{
    You can use Hartley’s F or Levene’s test.
}

\answerbreakline \setcounter{section}{4} \setcounter{answer}{0}

\answer{
    The observations are not independent because the same twelve people are tested twice. The two blood pressure measurements for one person are connected/dependent: a person with high blood pressure will have higher values in both experiments. That is why in a dependent t-test you look at the difference between the two measurements.
}

\clearpage % Page break

\answer{
Mean standing: 137.07 \\
Mean sitting:  145.43 \\
Mean difference: 8.36 \\
\\
\answercode{{\color{dataset}\# These are the values for the blood pressure data set} \\
standing <- c(136, 144, 152, 133, 140, 129, 131, 133, 145, 134, 142, 140, 125, 135) \\
sitting <- c(148, 159, 121, 151, 145, 139, 135, 144, 141, 148, 143, 161, 150, 151) \\
 \\
mean(standing)   {\color{dataset}\# Mean standing: 137.07} \\
mean(sitting)      {\color{dataset}\# Mean sitting: 145.43} \\
differences <- sitting - standing \\
mean(differences) {\color{dataset}\# Mean difference: 8.36}}
}

\answer{
    $H_0$: $\mu_D \leq 0$ \hspace*{4cm} $H_1$: $\mu_D > 0$
}

\answer{
    The test is done with \rcode{alternative = \textquotesingle greater\textquotesingle}, which means that now \texttt{R} will test for standing greater than sitting, which is quite improbable given the sample results. \\
    \\
    \answercode{t.test(x = standing, y = sitting, alternative = \textquotesingle greater\textquotesingle, mu = 0, \\
\hspace*{40pt}paired = TRUE, conf.level = 0.925)}
}

\answer{
    The p-value for this sample outcome is 0.02055, which is below the limit of 0.075 (92.5\% confidence). $H_0$ is rejected with 92.5\% confidence. The blood pressure is significantly higher lying down than standing up. There is a risk of 7.5\% for type-I error. \\
    \\
    \answercode{{\color{dataset}\# Correct: alternative: x = sitting, y = standing} \\
    t.test(x = sitting, y = standing, alternative = \textquotesingle greater\textquotesingle, mu = 0, \\
\hspace*{40pt}paired = TRUE, conf.level = 0.925)}
}

\clearpage % Page break