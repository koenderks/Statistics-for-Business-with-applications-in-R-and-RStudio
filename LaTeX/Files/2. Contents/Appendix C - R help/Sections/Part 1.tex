\section{Part \RomanNumeralCaps{1}: Basic functionality}

\textit{1.1 Types of data} \\
\\
\begin{minipage}[t]{.4\textwidth}
\vspace*{-8pt}
\rcode{numeric} \\
\rcode{character} \\
\rcode{logical} \\
\rcode{factor} \\ 			
\rcode{Inf} \\					
\rcode{NaN} \\ 					 
\rcode{NA}  
\end{minipage}
\begin{minipage}[t]{.6\textwidth}
Numbers \\
Words (text) \\
TRUE or FALSE \\
One of the above with predefined categories \\
Infinite \\
Not a number \\
Not available
\end{minipage}
\vspace*{.5cm}

\textit{1.2 Assigning values to variables} \\
\\
There are multiple ways to assign a value to a variable. For example, these all do the same, which is assigning the value \rcode{1} to the variable \rcode{a}.\\
\\
\begin{center}
    \rcode{a = 1} or \rcode{a <- 1} \hspace{2.5cm} \rcode{assign(\textquotesingle a\textquotesingle, 1)}  \hspace{2.5cm}	\rcode{a = b = 1} 
\end{center}
\vspace*{0.5cm}
\textit{1.3 Data structures} \\
\\
\begin{minipage}[t]{.4\textwidth}
\vspace*{-8pt}
\rcode{vector} \\ 				 
\rcode{matrix} \\				
\rcode{array} 	\\			
\rcode{data frame} \\				 
\rcode{list} 				
\end{minipage}
\begin{minipage}[t]{.6\textwidth}
A one-dimensional data structure \\
A two-dimensional data structure \\
A multi-dimensional data structure \\
A data structure containing different types of data \\
A collection of different data structures
\end{minipage}
\vspace*{.5cm}

\textit{1.4 Vectors} \\
\\
\begin{minipage}[t]{.4\textwidth}
\vspace*{-8pt}
\rcode{c(1, 2, 3)} \\			 
\rcode{seq(1, 6, 2) } \\			
\rcode{rep(1:3, 2)} \\				
\rcode{1:4} \\ 					
\rcode{paste(x, y)} \\ 				 
\rcode{letters[1:5]} \\ 			
\rcode{sample(x, 5)} \\ 			 
\rcode{length(x) } \\				
\rcode{cut(x, 5)} \\ 				
\rcode{append(x, c(4, 5))} \\ 		
\rcode{x <- numeric()} \\ 			
\rcode{sort(x) 	} \\			
\end{minipage}
\begin{minipage}[t]{.6\textwidth}
Combine the numbers 1, 2, and 3 in a vector \\
Create sequence from 1 to 6 in increments of 2 \\
Repeat 1 to 3, and do that 2 times \\
Vector of 1 t/m 4 (\rcode{:} is therefore making a vector) \\
Paste vectors \rcode{x} and y together \\
Vector of first 5 letters of the alphabet  \\
Gives a random sample of size 5 from of data \rcode{x} \\
Indicates the length of \rcode{x} \\
Divide \rcode{x} in vectors with length 5 \\
Add the numbers 4 and 5 to vector \rcode{x} \\
Creates an empty vector \rcode{x} \\
Order vector \rcode{x} from low to high (default) \\
\end{minipage}
\vspace*{.5cm}

\clearpage % Page break

\textit{1.5 Matrices} \\
\\
\begin{minipage}[t]{.4\textwidth}
\vspace*{-8pt}
\rcode{matrix(1:9, 3, 3) } \\			
\rcode{diag(x) 		} \\		
\rcode{head(x, 2) } \\				
\rcode{t(x) 	} \\			
\rcode{rbind(x, y) 	} \\		
\rcode{cbind(x, y) 	} \\		
\end{minipage}
\begin{minipage}[t]{.6\textwidth}
Create a matrix of 1 t/m 9 with 3 rows and 3 columns  \\
Get the diagonal from matrix \rcode{x} \\
Gives the first two rows of matrix or data frame \rcode{x} \\
Gives transpose of matrix \rcode{x} \\
Add rows from matrix \rcode{x} and \rcode{y} together \\
Add columns from matrix \rcode{x} and \rcode{y} together \\
\end{minipage}
\vspace*{.5cm}

\textit{1.6 Data frames} \\
\\
\begin{minipage}[t]{.4\textwidth}
\vspace*{-8pt}
\rcode{data.frame(\textquotesingle X\textquotesingle = x) } \\
\rcode{head(x, 2) 	} \\			
\rcode{names(x) } \\				
\rcode{colnames(x) 	} \\			
\rcode{rownames(x) 	} \\
\end{minipage}
\begin{minipage}[t]{.6\textwidth}
Create a data frame with data \rcode{x} (\rcode{X} is the column name) \\
Look at first two rows of data frame \rcode{x} \\
Names of data frame \rcode{x} \\
Column names of data frame \rcode{x} \\
Row names of data frame \rcode{x} \\
\end{minipage}
\vspace*{.5cm}

\textit{1.7 Lists} \\
\\
\begin{minipage}[t]{.4\textwidth}
\vspace*{-8pt}
\rcode{x <- list()} \\ 				
\rcode{x[[\textquotesingle title\textquotesingle]] <- m} \\	
\end{minipage}
\begin{minipage}[t]{.6\textwidth}
Create an empty list \rcode{x}  \\
Insert structure \rcode{m} into list \rcode{x} (\rcode{title} is the new title)
\end{minipage}
\vspace*{.5cm}

\textit{1.8 Indexing} \\
\\
\begin{minipage}[t]{.4\textwidth}
\vspace*{-8pt}
\rcode{x[2] 	} \\				
\rcode{x[1:5] 	} \\			
\rcode{x[-1] 	} \\			
\rcode{x[x > 5] } \\				
\rcode{x[x > 3 \& x < 6] } \\			 
\rcode{x[1:3, 1] 	} \\		
\rcode{x[1:3, 1:4] 	} \\
\rcode{x\$h or x[\textquotesingle h\textquotesingle]	}	
\end{minipage}
\begin{minipage}[t]{.6\textwidth}
Get the second element from the vector \rcode{x} \\
Get the first to the fifth element from vector \rcode{x} \\
Get all elements except first element from vector \rcode{x} \\
Get all elements greater than 5 from vector \rcode{x} \\
Get all values from \rcode{x} greater than 3 and less than 6 \\
Get first 3 values from the first column from data \rcode{x} \\
Get first 3 values from the first four columns from \rcode{x} \\
Select element \rcode{h} from data frame \rcode{x}
\end{minipage}
\vspace*{.5cm}

\textit{1.9 Operators} \\
\\
\begin{minipage}[t]{.4\textwidth}
\vspace*{-8pt}
\rcode{x == y 	} \\			
\rcode{x != y } \\				
\rcode{\%\% } 	
\end{minipage}
\begin{minipage}[t]{.6\textwidth}
Check if \rcode{x} equals \rcode{y} \\
Check if \rcode{x} does not equal \rcode{y} \\
The remainder of a division (e.g. \rcode{36\%\%5} = 1)	
\end{minipage}
\vspace*{.5cm}

\textit{1.10 Importing data and files} \\
\\
\begin{minipage}[t]{.4\textwidth}
\vspace*{-8pt}
\rcode{data(\textquotesingle x\textquotesingle)} \\ 				
\rcode{file.choose() } \\ 			
\rcode{read.table(\textquotesingle x\textquotesingle)} \\  			
\rcode{write.table(x) } \\ 		
\rcode{read.csv(\textquotesingle x\textquotesingle) } \\ 		
\rcode{write.csv(x) } \\ 		
\end{minipage}
\begin{minipage}[t]{.6\textwidth}
Import data set \rcode{x} \\
Get access to interface to select a file \\
Read in a \dataset{.txt} file \rcode{x} from the working directory \\
Writes data \rcode{x} to a \dataset{.txt} file from the working directory \\
Reads in a \dataset{.csv} file \rcode{x} from the working directory \\
Writes data \rcode{x} to a \dataset{.csv} file from the working directory 
\end{minipage}
\vspace*{.5cm}

\clearpage % Page break

\textit{1.11 Basic functions} \\
\\
\begin{minipage}[t]{.4\textwidth}
\vspace*{-8pt}
\rcode{TAB} \\ 					
\rcode{ls() } \\ 					 
\rcode{rm(list = ls())} \\  		
\rcode{getwd()	} \\ 	 		
\rcode{setwd() 	} \\ 		
\rcode{help(x) 	} \\ 
\rcode{str(x)	 } \\ 		
\rcode{summary(x) 	} \\ 			
\rcode{print(\textquotesingle Hello\textquotesingle) } \\ 		
\rcode{round(x, digits = 2)} \\  		  
\rcode{which.max(x) } \\ 			 
\rcode{which(x == 10) 	} \\ 	
\rcode{unique(x)	} \\ 		
\rcode{length(x)	} \\ 		
\rcode{nrow(x)	} \\ 		
\rcode{ncol(x)	} \\ 				
\end{minipage}
\begin{minipage}[t]{.6\textwidth}
Scrolling through functions beginning with that letter  \\
See all variables available in the environment \\
Delete all variables in your environment \\
See the location of your working directory  \\
Set the location of your working directory  \\
Read help about the function \rcode{x} \\
Finds out the structure of data \rcode{x}  \\
Gives summary of the object \rcode{x} \\
Print \rcode{Hello} to the output \\
Round number(s) in \rcode{x} to a specified number of digits \\
Indicates the place of the highest value of data \rcode{x} \\
Shows the place of each object in \rcode{x} that equals 10 \\
Gives only the unique values in \rcode{x} \\
Gives the number of elements in \rcode{x} \\
Gives number of rows of matrix/data frame \rcode{x} \\
Gives number of columns of matrix/data frame \rcode{x}
\end{minipage}
\vspace*{.5cm}

\textit{1.12 Installing an add-on package} \\
\\
\begin{minipage}[t]{.4\textwidth}
\vspace*{-8pt}
\rcode{install.packages(\textquotesingle x\textquotesingle)}		
\end{minipage}
\begin{minipage}[t]{.6\textwidth}
Installs package with name \rcode{x} 	
\end{minipage}
\vspace*{.5cm}

\clearpage % Page break