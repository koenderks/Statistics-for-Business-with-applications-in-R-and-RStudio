\setcounter{section}{5}
\setcounter{subsection}{4}
\setcounter{question}{0}

%%%%%%%%%%%%%%%%%%%%%%%%%%%%%%%%%%%%%%%%%%%%%%%%%%%%%%%%%%%%%%%%%%%%%%%%%%%
% Assignment 5.4: Dependent two-sample t-test in R
%%%%%%%%%%%%%%%%%%%%%%%%%%%%%%%%%%%%%%%%%%%%%%%%%%%%%%%%%%%%%%%%%%%%%%%%%%%

\rassignment{Dependent two-sample t-test in R}

To test for a difference in blood pressure depending on their position, 12 people’s blood pressure was measured twice, first standing up and then lying down. Researchers want to use this experiment to show with 92.5\% confidence that the blood pressure lying down is significantly higher than standing up. They are going to evaluate the results by comparing the \concept{mean} using a \concept{two-sample t-test}. \\

\question{
    Why is this a dependent (also called paired) \concept{t-test}?
}

\twolineanswerbox

Run the following code in \texttt{R}: \\

\codeblock{{\color{dataset}\# These are the values for the blood pressure data set} \\
standing <- c(132, 146, 135, 141, 139, 162, 128, 137, 145, 151, 131, 
              143) \\
lying <- c(136, 145, 140, 147, 142, 160, 137, 136, 149, 158, 120, 150)}

\question{
    Calculate the \concept{mean} of the \rcode{standing} and \rcode{lying} data sets, also calculate the differences and the \concept{mean} of the differences.
}

\rcodeanswersmall

\emptyanswerbox{
    Mean standing: \quad \shortanswerline
    \answerskip
    Mean lying: \qquad \quad \shortanswerline
    \answerskip
    Mean difference: \shortanswerline
}

\clearpage % Page break

\question{
    Write down the \concept{null hypothesis} $H_0$ and \concept{alternative hypothesis} $H_1$ for a one-sided test where the researchers want to show with 92.5\% confidence that the \concept{mean} blood pressure is higher lying down than standing up.
}

\hypothesesbox

Run the following code in \texttt{R}: \\

\codeblock{t.test(x = standing, y = lying, alternative = \textquotesingle greater\textquotesingle, mu = 0 , \\
       \hspace*{40pt}paired = TRUE, conf.level = 0.925)}
       
\question{
    Check the results of this test. The outcome is a bit strange, can you explain what is wrong here?
}

\onelineanswerbox

\question{
    Fix the \texttt{R} code to do a correct test and draw the correct (four part) conclusion.
}

\rcodeanswersmall

\sixlineanswerbox

\clearpage % Page break